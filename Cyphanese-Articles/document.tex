\documentclass{jsarticle}

\title{構文厳密な日本語話者のための暗号言語(仮称:Cyphanese)}
\author{秋葉 沙都季}
\date{\today}

\begin{document}
    \maketitle
    \begin{abstract}
        本論考では構文厳密性を保ち、日本語話者が容易に習得可能でありながら、暗号を直感的に扱えるセキュアな暗号言語の外観を示す。
        
        また、言語の暗号化に用いられるアルゴリズムや、構文解析器等はHaskellによって実装される。高校数学程度の数学的知識、日本語文法、および様相述語論理等の知識を要求するが臨機に説明を加える。
    \end{abstract}
    
    \section{Cyphaneseの特徴と概観}
    Cyphaneseは構文厳密性を保つ暗号言語であり、日本語話者の学習容易さを一つの指標として掲げている。
    目的の一つは日本語話者間のセキュアかつ厳密な情報伝達を可能にし、かつ容易にそれを用いてコミュニケーションを図るためである。
    また、Cyphaneseは特定の一言語を指す名称ではなく、この文法システムと理論を用いて生成された言語の総称である。
    この言語の話者は相手と事前に交換した鍵を用いることで言語を生成し、セキュアな情報伝達とコミュニケーションを図れる。
    
    この言語は日本語の単語を基盤としたa posterioriな言語であるが、暗号化を施すため都度単語の語形が変化することがある。
    そのため、日本語話者が容易に発話可能にするために、機械学習を用いて日本語らしい音韻的特徴を有する語形に変換するアルゴリズムを開発している。
    これはすきえんてぃあ氏(@cicada3301\_kig)のユーフォニー指数から着想を得ているが、結果としては大きく異なることを留意されたい。

    \section{構文厳密性}
    Cyphaneseは構文厳密性を有するが、構文厳密性を担保するためにPEGを用いる。


\end{document}